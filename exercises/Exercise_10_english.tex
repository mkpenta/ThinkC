% LaTeX source for textbook ``How to think like a computer scientist''
% Copyright (C) 1999  Allen B. Downey
% Copyright (C) 2009  Thomas Scheffler

%%%%%%%%%%%%%%%%%%%%%%%%%%%%%%%%%%%%%%%%%%

\begin{exercise}

Write a program that fills an integer array with 50 random number. Open a file called numbers.txt. Use a loop to print all the numbers to the file. Be sure to close the file.
\end{exercise}


%%%%%%%%%%%%%%%%%%%%%%%%%%%%%%%%%%%%%%%%%%

\begin{exercise}
Write a program that opens the numbers.txt file. Reads all 50 numbers to an array. Then prints all 50 numbers and their sum.
\end{exercise}

%%%%%%%%%%%%%%%%%%%%%%%%%%%%%%%%%%%%%%%%%%

\begin{exercise}
Consider the following struct:
\begin{verbatim}
	typedef struct
	{
		char initial1;
		char intial 2;
		char intial3;
		int score;
	} HighScoreEntry;
\end{verbatim}
Write a program that crates an array of 10 highs score structs and write them to a binary file called scores.dat.

Write the first operation of this program is to open the scores file and load the values in an array of HighScore structs. Next the program that display a menu to the user to 1)display scores 2) updates score 3)quit. Display scores print the struct data values in the array. Update score allows the user update any score in the array by giving the index and the new struct values. Quit will  rewrite the file to reflect the changes in the array. Use a loop  allow the user to run the menu as often as they like. Quit the progam and rerun it, your score changes should persist.

\end{exercise}


