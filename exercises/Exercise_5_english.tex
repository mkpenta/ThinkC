% LaTeX source for textbook ``How to think like a computer scientist''
% Copyright (C) 1999  Allen B. Downey
% Copyright (C) 2009  Thomas Scheffler


\begin{exercise}
If you are given three sticks, you may or may not be able to arrange
them in a triangle.  For example, if one of the sticks is 12 inches
long and the other two are one inch long, it is clear that you will
not be able to get the short sticks to meet in the middle.  For any
three lengths, there is a simple test to see if it is possible to form
a triangle:

\begin{quotation}
``If any of the three lengths is greater than the sum of the other two,
then you cannot form a triangle.  Otherwise, you can.''
\end{quotation}

Write a function named {\tt isTriangle()} that it takes three integers as
arguments, and that returns either {\tt TRUE} or {\tt FALSE},
depending on whether you can or cannot form a triangle from sticks
with the given lengths.


The point of this exercise is to use conditional statements to
write a function that returns a value.
\end{exercise}


%%%%%%%%%%%%%%%%%%%%%%%%%%%%%%%%%%%%


%%%%%%%%%%%%%%%%%%%%%%%%%%%%%%%%%%%%

\begin{exercise}
What is the output of the following program?  Is there any dead code in this program?

The purpose of this exercise is to make sure you understand logical operators and the flow of execution through fruitful methods.

\begin{verbatim}
  #define TRUE 1
  #define FALSE 0

  short isHoopy (int);
  short isFrabjuous (int);

  int main (void) 
  {
      short flag1 = IsHoopy (202);
      short flag2 = IsFrabjuous (202);
      printf ("%i\n", flag1);
      printf ("%i\n", flag2);
      if (flag1 && flag2) 
      {
          puts ("ping!");
      }
      if (flag1 || flag2) 
      {
          puts("pong!");
      }
      return EXIT_SUCCESS;
  }

short isHoopy (int x)
{
	short hoopyFlag;
	if (x%2 == 0) 
	{
		hoopyFlag = TRUE;
	} 
	else 
	{
		hoopyFlag = FALSE;
	}
	return hoopyFlag;
}

short isFrabjuous (int x) 
{
	short frabjuousFlag;
	if (x > 0) 
	{
		frabjuousFlag = TRUE;
	}
	else 
	{
		frabjuousFlag = FALSE;
	}
	return frabjuousFlag;
}

  
\end{verbatim}
\end{exercise}



\begin{exercise}
\begin{enumerate}


\item Create a new program called {\tt Sum.c},
and type in the following two functions, their prototypes a main function.

\begin{verbatim}
  int functionOne (int m, int n) 
  {
      if (m == n) 
      {
          return n;
      } 
      else 
      {
          return m + functionOne (m+1, n);
      }
  }

  int functionTwo (int m, int n) 
  {
      if (m == n) 
      {
          return n;
      } 
      else 
      {
          return n * functionTwo (m, n-1);
      }
  }
\end{verbatim}
%
\item Write a few lines in {\tt main()} to test these functions.
Invoke them a couple of times, with a few different values,
and see what you get.  By some combination of testing and
examination of the code, figure out what these functions do,
and give them more meaningful names.  Add comments that
describe their function abstractly.

\item Add a {\tt prinf} statement to the beginning of both
functions so that they print their arguments each time they are
invoked.  This is a useful technique for debugging recursive
programs, since it demonstrates the flow of execution.

\end{enumerate}
\end{exercise}

\begin{exercise}
\label{ex.power}
Write a recursive function called {\tt power()} that
takes a double {\tt x} and an integer {\tt n} and that
returns $x^n$.  

Hint: a recursive definition of this
operation is {\tt power (x, n) = x * power (x, n-1)}.
Also, remember that anything raised to the zeroeth power
is 1.
\end{exercise}


%

\begin{exercise}
The distance between two points $(x_1, y_1)$ and $(x_2, y_2)$
is

\[Distance = \sqrt{(x_2 - x_1)^2 + (y_2 - y_1)^2} \]

Please write a function named {\tt distance()} that takes four
doubles as parameters---{\tt x1}, {\tt y1}, {\tt x2} and {\tt
y2}---and that prints the distance between the points.

You should first write a function called  {\tt sumSquares()}
that calculates and returns the sum of the squares of its arguments.
For example:

\begin{verbatim}
    double x = sumSquares (3.0, 4.0);
\end{verbatim}
%
would assign the value {\tt 25.0} to {\tt x}.

The point of this exercise is to write a new function that uses an
existing one.  You should first write the sumSquares function then use that function in your distance function.  Write a main function that tests each function
\end{exercise}


\begin{exercise}
The point of this exercise is to practice the syntax of fruitful
functions.

\begin{enumerate}

\item Use your existing solution to Exercise~\ref{ex.multadd} and make sure
you can still compile and run it.

\item Transform {\tt Multadd()} into a fruitful function, so
that instead of printing a result, it returns it.

\item Everywhere in the program that {\tt Multadd()} gets
invoked, change the invocation so that it stores the
result in a variable and/or prints the result.

\end{enumerate}
\end{exercise}

