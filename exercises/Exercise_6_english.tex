% LaTeX source for textbook ``How to think like a computer scientist''
% Copyright (C) 1999  Allen B. Downey
% Copyright (C) 2009  Thomas Scheffler

%%%%%%%%%%%%%%%%%%%%%%%%%%%%%%%%%%%%%%

\begin{exercise}\label{infloop}
%changed the condition on the loop so that it will terminate
%(was this *supposed* to be an infinite loop?)
\begin{verbatim}
    void loop(int n) 
    {
        int i = n;
        while (i > 1) 
        {
            printf ("%i\n",i);
            if (i%2 == 0) 
            {
                i = i/2;
            } 
            else 
            {
                i = i+1;
            }
        }
    }

    int main (void) 
    {
        loop(10);
        return EXIT_SUCCESS;
    }
\end{verbatim}
%
\begin{enumerate}

\item  Draw a table that shows the value of the variables {\tt i} and {\tt n} during the execution of the program. 
The table should contain one column for each variable and one line for each iteration.


\item What is the output of this program?

\end{enumerate}
\end{exercise}

%%%%%%%%%%%%%%%%%%%%%%%%%%%%%%%%%%%%%%


\begin{exercise}
In Exercise~\ref{ex.power} we wrote a recursive version of {\tt
power()}, which takes a double {\tt x} and an integer {\tt n} and
returns $x^n$.  Now write an iterative function to perform the same
calculation.
\end{exercise}

%%%%%%%%%%%%%%%%%%%%%%%%%%%%%%%%%%%%%%



\begin{exercise}
	In Exercise~\ref{ex.power} we wrote a recursive version of {\tt
		power()}, which takes a double {\tt x} and an integer {\tt n} and
	returns $x^n$.  Now write an iterative function to perform the same
	calculation.
\end{exercise}

%%%%%%%%%%%%%%%%%%%%%%%%%%%%%%%%%%%%%%

