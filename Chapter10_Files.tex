% LaTeX source for textbook ``How to think like a computer scientist''
% Copyright (C) 1999  Allen B. Downey
% Copyright (C) 2009  Thomas Scheffler
% Copyright (C) 2023 Michael K Penta

\chapter{Working with files}
\label{files}
\index{files}


\section{Introduction}
File handling is a fundamental aspect of programming, allowing us to store and retrieve
data that can be used across multiple runs of a program. In C programming, file handling
is achieved using the stdio.h library, which provides a variety of functions for reading
from and writing to files. In this chapter, we will explore the basics of file handling
in C, including opening and closing files, reading and writing data to files, and
handling errors that may occur during file input/output.

\section{Text Files vs Binary Files}
Before diving into working with files in C, it is important to understand the differences
between text files and binary files. Text files are files that contain human-readable
text, such as plain text documents or CSV files. These files are composed of a series of
characters, usually encoded in ASCII or Unicode format. Binary files, on the other hand,
are files that contain machine-readable data, such as executable files or images. These
files are composed of a series of bytes that represent binary data and are not
human-readable.

The choice between text files and binary files depends on the type of data being stored
and the intended use of the file. Text files are usually used for storing data that can
be easily edited by humans, such as configuration files or log files. On the other hand,
binary files are typically used for storing data that is not easily editable by humans,
such as multimedia files or database files.

One of the key differences between text files and binary files is the way that they are
read and written. Text files can be read and written using standard functions such as
{\tt fgets()} and {\tt fprintf()}, while binary files require more specialized functions
such as {\tt fread()} and {\tt fwrite()}.

\section{Opening and Closing Files}
Once we have decided on the type of file we want to use, the next step is to open the
file. In C, we use the {\tt fopen()} function to open a file, which takes two parameters:
the name of the file to be opened and the mode in which it is to be opened. The mode
parameter specifies whether the file is to be opened for reading, writing, or both, as
well as whether the file is to be created if it does not exist.

There are several modes in which a file can be opened, including:
\begin{description}
\item[r]: Open file for reading
\item[w]: Open file for writing (overwrite if file already exists)
\item[a]: Open file for appending (write to end of file)
\item[r$+$]: Open file for both reading and writing
\item[w$+$]: Open file for both reading and writing (overwrite if file already exists)
\item[a$+$]: Open file for both reading and writing (write to end of file)
\end{description}
Once a file has been opened, we can perform operations on it, such as reading data from
or writing data to the file. When we are finished working with a file, it is important to
close it using the {\tt fclose()} function. This ensures that any changes made to the
file are saved and any resources used by the file are freed up.


\section{Reading from Text Files}
Imagine we have a text file called "students.txt" containing a list of students' names, one name per line. We want to read the data from this file and display it on the screen. The following example demonstrates how to read data from a text file using {\tt fgets()}:

\begin{verbatim}
	FILE *file = fopen("students.txt", "r");
	if (file != NULL) 
	{
		char name[256];
		printf("List of Students:\n");
		while (fgets(name, sizeof(name), file) != NULL) {
			printf("%s", name);
		}
		fclose(file);
	}
\end{verbatim}

\section{Writing to Text Files}
Suppose we want to create a text file called "grades.txt" to store the grades of students in a course. We have a struct called "Student" containing the student's name and grade. The following example demonstrates how to write data to a text file using {\tt fprintf()}:

\begin{verbatim}
	typedef struct 
	{
		char name[50];
		int grade;
	} Student;
	
	Student students[] =
	 {
		{"Alice", 85},
		{"Bob", 92},
		{"Carol", 78}
	};
	
	FILE *file = fopen("grades.txt", "w");
	if (file != NULL)
	 {
		for (int i = 0; i < 3; i++) 
		{
			fprintf(file, "Name: %s, Grade: %d\n", students[i].name, students[i].grade);
		}
		fclose(file);
	}
\end{verbatim}

\section{Reading from Binary Files}
Consider a binary file called "inventory.bin" containing a list of products in a store. 
Each product has a struct called "Product" containing a product ID and its stock quantity.
 The following example demonstrates how to read data from a binary file using {\tt fread()}:

\begin{verbatim}
	typedef struct 
	{
		int productID;
		int stock;
	} Product;
	
	FILE *file = fopen("inventory.bin", "rb");
	if (file != NULL)
	 {
		Product products[5];
		size_t read_elements = fread(products, sizeof(Product), 5, file);
		
		printf("Product Inventory:\n");
		for (int i = 0; i < read_elements; i++)
		{
			printf("Product ID: %d, Stock: %d\n", products[i].productID, products[i].stock);
		}
		
		fclose(file);
	}
\end{verbatim}

\section{Writing to Binary Files}
Suppose we want to create a binary file called "sales.bin" to store the sales data of a store. Each sale record has a struct called "Sale" containing a transaction ID, product ID, and quantity sold. The following example demonstrates how to write data to a binary file using {\tt fwrite()}:

\begin{verbatim}
	typedef struct {
		int transactionID;
		int productID;
		int quantity;
	} Sale;
	
	Sale sales[] =
	 {
		{1, 101, 5},
		{2, 102, 3},
		{3, 103, 8}
	};
	
	FILE *file = fopen("sales.bin", "wb");
	if (file != NULL) 
	{
		fwrite(sales, sizeof(Sale), 3, file);
		fclose(file);
	}
\end{verbatim}


\section{File Pointers and Positioning}
File pointers are an important concept in C file handling, as they allow us to
keep track of our current position within a file. A file pointer is a variable
that holds the position of the next byte to be read or written in a file. By
manipulating the file pointer, we can control where in the file we are reading
or writing data.

The {\tt ftell()} function is used to determine the current position of the file
pointer within a file. It takes the file pointer as an argument and returns the
current position in bytes from the beginning of the file.

The {\tt fseek()} function is used to move the file pointer to a specific
position within a file. It takes three arguments: the file pointer, the offset
(number of bytes to move the pointer), and the origin (starting position for
the pointer movement). The origin can be one of the following:

\begin{itemize}
	\item {\tt SEEK$\_$SET}: Beginning of file
	\item {\tt SEEK$\_$CUR}: Current position of file pointer
	\item {\tt SEEK$\_$END}: End of file
\end{itemize}

For example, to move the file pointer to the beginning of the file, we would use:

\begin{verbatim}
	fseek(file_pointer, 0, SEEK_SET);
\end{verbatim}

The {\tt rewind()} function is a simpler way to move the file pointer to the
beginning of the file. It takes the file pointer as an argument and moves the
pointer to the beginning of the file, equivalent to using {\tt fseek()} with
an offset of 0 and {\tt SEEK$\_$SET} as the origin.

When working with files, it is important to be aware of common errors that can
occur. One common error is failing to check if a file was opened successfully
before attempting to read or write to it. This can lead to unexpected behavior
or even crashes. Another common error is not properly closing a file when we
are finished using it. This can lead to resource leaks and cause issues if we
attempt to open the file again later.

\section{Finding the size of a file}

Determining the size of a file can be useful in many scenarios, such as when reading or writing binary data to a file. In C, we can use the {\tt ftell()} and {\tt fseek()} functions to determine the size of a file.

The {\tt ftell()} function returns the current position of the file pointer within a file. We can use this function to determine the size of a file by first moving the file pointer to the end of the file using {\tt fseek()}, and then calling {\tt ftell()} to get the position of the file pointer.

\section{Glossary}

\begin{description}
	\item[File handling:] The process of reading from and writing to files in a program.
	\item[Text file:] A file that contains human-readable text, such as plain text documents or CSV files.
	\item[Binary file:] A file that contains machine-readable data, such as executable files or images.
	\item[ASCII:] A character encoding standard that represents each character as a unique 7-bit code.
	\item[Unicode:] A character encoding standard that represents each character as a unique 16-bit or 32-bit code.
	\item[Mode:] A parameter used when opening a file that specifies the intended use of the file (e.g. read, write, append).
	\item[Error handling:] The process of handling errors that may occur during file input/output.
	\item[File pointer:] A variable that points to a specific position in a file.
\end{description}

\section{Exercises}
\setcounter{exercisenum}{0}

% LaTeX source for textbook ``How to think like a computer scientist''
% Copyright (C) 1999  Allen B. Downey
% Copyright (C) 2009  Thomas Scheffler

%%%%%%%%%%%%%%%%%%%%%%%%%%%%%%%%%%%%%%%%%%

\begin{exercise}

Section~\ref{Structures as parameters} defines the function {\tt PrintPoint()}. 
The argument of this function is passed along as a value (\emph{call-by-value}).

\begin{enumerate}
\item Change the definition of this function, so that it only passes a reference to 
the structure to the function for printing (\emph{call-by-reference}).

\item Test the new function with different values and document the results. 
\end{enumerate}


\end{exercise}


%%%%%%%%%%%%%%%%%%%%%%%%%%%%%%%%%%%%%%%%%%

\begin{exercise}

Write a program, that defines a struct \texttt{Person\_t}.
This struct should contain two members. The first member should be a string
of sufficient length, to contain the name of a person. The second member should
be an integer value containing the persons age. 

\begin{enumerate}
\item Define two struct variables \texttt{person1} and \texttt{person2}.\\ 
Initialize the two struct variables with suitable values. The first person
should be called Betsy. Betsy should be 42 years old.\\
The second person should be named after yourself and should also be as old as yourself.


\item Write a function \texttt{PrintPerson()}. The function should take a struct variable of type
\texttt{Person\_t} as an argument and print the name of the person and the corresponding age.
\item Write a function \texttt{HappyBirthday()}. The function should increase the age of the 
corresponding person by one year and print out a birthday greeting.\\
\textbf{Hint: }You need to passes a reference to the structure to the function (\emph{call-by-reference}). 
If your function uses \emph{call-by-value}, the age only changes in the local copy of the struct and
the changes you have made are lost, once the function terminates. Try it out, if you do not believe me!
\end{enumerate}


\end{exercise}






